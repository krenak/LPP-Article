\begin{onehalfspace}
\footnotesize
    \noindent A energia elétrica é um recurso essencial para o desenvolvimento econômico de 
    um país, para a qualidade de vida da população e para a manutenção do meio ambiente
    por meio de seu uso eficiente. A importância do uso racional e eficiente deste
    recurso torna imprescindível a conservação e redução do seu desperdício para a 
    sustentabilidade do ambiente em que se vive. Esta gestão eficiente do consumo de energia
    é essencial para reduzir o impacto energético de setores como o de edificações, 
    o qual consome cerca de 36 a 40\% da energia total final no mundo.\vspace*{0.3cm} \newline
    \noindent Um exemplo da importância deste recurso e de seu uso devidamente planejado 
    pôde ser observado durante a crise brasileira. Ocorrida em 2001, provocou mudanças 
    no planejamento do fornecimento de energia elétrica, com o posterior surgimento de 
    medidas atenuantes às dificuldades de cunho ambiental e de infraestrutura da época. 
    Em seu ápice, no ano de 1999, o país passou pelo período popularmente denominado 
    “apagão”, o qual representou a falta de fornecimento em 70\% do território nacional. 
    O consumo de energia elétrica, entre os anos de 1990 e 2000, sofreu aumento de 
    49\%, enquanto a capacidade instalada foi expandida em 35\%, ocasionando o 
    descompasso entre consumo e fornecimento nesta época.\vspace*{0.3cm} \newline
    \noindent No âmbito estadual, o Espírito Santo vem apresentando redução na produção 
    de energia limpa quando comparado proporcionalmente ao consumo de fontes tradicionais. 
    Existe ainda a parcela de geração de energia elétrica oriunda de fontes não-renováveis 
    de energia, como usinas termelétricas, correspondendo a 65\% de toda a capacidade 
    instalada em operação do Espírito Santo, restando 35\% de fontes renováveis, composta 
    por usinas hidrelétricas, com participação de 34\%, e geradores de energia solar 
    fotovoltaica, com 1\%.\vspace*{0.3cm} \newline
    \noindent Em contraponto à demanda e ineficiência energética, as edificações
    comerciais, em particular as de escritório, podem desempenhar funções estratégicas 
    como minimizar o uso energético e produzir eletricidade, aproximando ou tornando 
    zero a razão entre a produção e o consumo de energia. Estas edificações são 
    denominadas edificações com balanço energético nulo, 
    ou \textit{Zero Energy Buildings} – ZEB.
    Com a introdução de uma ZEB, a exploração de recursos renováveis complementares 
    como a energia solar, e a utilização de tecnologia solar fotovoltaica, surgem como 
    opção para minimizar as consequências negativas causadas por condições climáticas, 
    de infraestrutura e socioeconômicas adversas.\vspace*{0.3cm} \newline
    \noindent A disponibilidade de recursos naturais como a radiação solar recebida 
    no Brasil, por exemplo, concentra grande capacidade de geração de energia solar,
    alcançando a ordem de 1.013 MWh. Este nível de radiação nível acima de países 
    tradicionais na geração de energia fotovoltaica, o que qualidfica a adoção deste 
    recurso como forma de reduzir o uso de fontes de energia fósseis e como economia 
    no consumo de água.\vspace*{0.3cm} \newline
    \noindent Considerando as características do ambiente construído no âmbito da 
    Região Metropolitana da Grande Vitória, é possível desenvolver edificações cujos 
    valores de demanda e produção de energia elétrica resultem em nulo ou quase nulo? 
    O objetivo principal desta pesquisa foi avaliar a aplicabilidade do conceito 
    \textit{Zero Energy} em edificações comerciais, especificamente de escritório, 
    tomando como local de estudo de caso o município de Vitória (ES).\vspace*{0.3cm} \newline
    \noindent Assim, como ponto de partida, define-se que um edifício 
    \textit{Zero Energy} – ZEB, ou em português, balanço energético nulo, é uma 
    edificação energeticamente eficiente onde, considerada a fonte energética, a 
    energia elétrica fornecida pela concessionária é anualmente menor ou igual à 
    quantidade de energia renovável exportada pela edificação para a rede.\vspace*{0.3cm} \newline
    \noindent Uma outra definição importante iniciada na Europa foi proposta para 
    edificações \textit{Near Net Zero Energy}, nZEB, e em português, próximo ao 
    balanço energético nulo, se apoia na premissa do aproveitamento máximo 
    de recursos para produção de energia, implementando mecanismos à edificação de 
    forma que este aproveitamento aconteça, e a utilização à nível ótimo da energia 
    primária, para um consumo maior que 0 kWh/m² ao ano.\vspace*{0.3cm} \newline
    \noindent Estudos desenvolvidos pela \textit{United Nations Enviroment Programme}
    apontam que 103 países definiram a eficiência energética e uso de energias 
    renováveis como parte importante do seu planejamento estratégico, e destes, 
    79 são países emergentes e em desenvolvimento. Constata-se, ainda, que o 
    consumo de energia poderia ter sido 12\% maior em 2017 caso as políticas 
    públicas mencionadas anteriormente não tivessem sido implementadas desde o ano 2000.\vspace*{0.3cm} \newline
    \noindent No Brasil, a taxa de consumo energético, assim como em outros países, 
    é definida pelo aquecimento econômico e cenários estabelecidos para o desenvolvimento 
    esperado para o país. Nesse sentido, espera-se que o Brasil, até 2026, apresente 
    crescimento econômico e, concomitantemente, consuma energia de forma modesta. 
    Projeta-se que este crescimento seja da ordem de 1,9\% ao ano até a metade da 
    década analisada, com variações que definem o crescimento do consumo em 2,3\% 
    anuais, indicando otimismo para o setor de energia brasileiro.\vspace*{0.3cm} \newline
\end{onehalfspace}
