\begin{abstract}
    \noindent O consumo de energia no uso de edificações vem crescendo gradativamente ao longo
    das últimas décadas, fruto do desenvolvimento industrial e da revolução tecnológica
    que vem acompanhando este movimento. A emissão de gases poluentes e a modificação 
    do clima são consequências desse cenário de desenvolvimento e consumo. Aliado a 
    esses fatores, as edificações contribuem para o agravamento  desse  cenário,  uma  
    vez  que  o  uso  destas  acarreta  em  impactos  negativos significativos ao meio 
    ambiente. Em contraponto, edificações energeticamente eficientes vêm se tornando 
    pré-requisito  para  o  planejamento  de  novos  ambientes  construídos,  
    modificando  a forma  como  a  comunidade  percebe  a  relação entre a edificação  
    e  o  consumo de  energia.  Este trabalho  tem  como  objetivo  estudar  o  
    potencial  de  aplicação  do  conceito Zero Energy  para edificações comerciais, 
    com o intuito de verificar a validade do método para o cenário construtivo 
    brasileiro adotando como estudo de caso  uma edificação  em Vitória (ES). 
    Metodologicamente, este  estudo  foi  desenvolvido  com  base  em  três  grandes  
    etapas,  onde  a  primeira  consistiu  em realizar  o  levantamento  das  
    edificações  dentro  de  um  recorte  territorial  pré-estabelecido, selecionar 
    as características construtivas e arquitetônicas mais frequentes entre elas e 
    construir modelos  representativos  do  cenário  observado;  a  segunda  consistiu  
    em  submeter  os  modelos representativos à simulações computacionais para avaliar 
    o desempenho energético, as possíveis formas  de  eficientização  e  de  produção  
    de  energia;  e  por  fim,  a  terceira  etapa,  na  qual  foi realizada  avaliação  
    dos  resultados  e  da  viabilidade  econômica  de  implantação  do  sistema  
    de produção de energia. Os resultados mostraram que as estratégias de implementação 
    de sistemas de condicionamento de ar, de equipamentos e iluminação mais eficientes 
    são muito importantes para  a  economia  de  energia.  É  perceptível  que  a  
    proposição  de  soluções  construtivas  e arquitetônicas  mais  eficientes  em  
    relação  ao  desempenho  energético  associado  a  técnicas  de obtenção  de  
    energia  podem  resultar  em  uma  edificação  com  o  balanço  energético  nulo  
    ou próximo ao nulo. Esses resultados indicam que a adoção desse conceito para novas 
    edificações é factível e cada vez mais acessível à comunidade.
    \paragraph{Palavras-chave:} zero energy buildings; balanço energético nulo; edifício de escritório 12 
\end{abstract}\pagebreak